% This is "sig-alternate.tex" V2.1 April 2013
% This file should be compiled with V2.5 of "sig-alternate.cls" May 2012
%
% This example file demonstrates the use of the 'sig-alternate.cls'
% V2.5 LaTeX2e document class file. It is for those submitting
% articles to ACM Conference Proceedings WHO DO NOT WISH TO
% STRICTLY ADHERE TO THE SIGS (PUBS-BOARD-ENDORSED) STYLE.
% The 'sig-alternate.cls' file will produce a similar-looking,
% albeit, 'tighter' paper resulting in, invariably, fewer pages.
%
% ----------------------------------------------------------------------------------------------------------------
% This .tex file (and associated .cls V2.5) produces:
%       1) The Permission Statement
%       2) The Conference (location) Info information
%       3) The Copyright Line with ACM data
%       4) NO page numbers
%
% as against the acm_proc_article-sp.cls file which
% DOES NOT produce 1) thru' 3) above.
%
% Using 'sig-alternate.cls' you have control, however, from within
% the source .tex file, over both the CopyrightYear
% (defaulted to 200X) and the ACM Copyright Data
% (defaulted to X-XXXXX-XX-X/XX/XX).
% e.g.
% \CopyrightYear{2007} will cause 2007 to appear in the copyright line.
% \crdata{0-12345-67-8/90/12} will cause 0-12345-67-8/90/12 to appear in the copyright line.
%
% ---------------------------------------------------------------------------------------------------------------
% This .tex source is an example which *does* use
% the .bib file (from which the .bbl file % is produced).
% REMEMBER HOWEVER: After having produced the .bbl file,
% and prior to final submission, you *NEED* to 'insert'
% your .bbl file into your source .tex file so as to provide
% ONE 'self-contained' source file.
%
% ================= IF YOU HAVE QUESTIONS =======================
% Questions regarding the SIGS styles, SIGS policies and
% procedures, Conferences etc. should be sent to
% Adrienne Griscti (griscti@acm.org)
%
% Technical questions _only_ to
% Gerald Murray (murray@hq.acm.org)
% ===============================================================
%
% For tracking purposes - this is V2.0 - May 2012

\documentclass{sig-alternate-05-2015}


\begin{document}

% Copyright
\setcopyright{acmcopyright}
%\setcopyright{acmlicensed}
%\setcopyright{rightsretained}
%\setcopyright{usgov}
%\setcopyright{usgovmixed}
%\setcopyright{cagov}
%\setcopyright{cagovmixed}


% DOI
%\doi{10.475/123_4}

% ISBN
%\isbn{123-4567-24-567/08/06}

%Conference
%\conferenceinfo{PLDI '13}{June 16--19, 2013, Seattle, WA, USA}

%\acmPrice{\$15.00}

%
% --- Author Metadata here ---
%\conferenceinfo{WOODSTOCK}{'97 El Paso, Texas USA}
%\CopyrightYear{2007} % Allows default copyright year (20XX) to be over-ridden - IF NEED BE.
%\crdata{0-12345-67-8/90/01}  % Allows default copyright data (0-89791-88-6/97/05) to be over-ridden - IF NEED BE.
% --- End of Author Metadata ---

\title{Title}
\subtitle{Subtitle}

% You need the command \numberofauthors to handle the 'placement
% and alignment' of the authors beneath the title.
%
% For aesthetic reasons, we recommend 'three authors at a time'
% i.e. three 'name/affiliation blocks' be placed beneath the title.
%
% NOTE: You are NOT restricted in how many 'rows' of
% "name/affiliations" may appear. We just ask that you restrict
% the number of 'columns' to three.
%
% Because of the available 'opening page real-estate'
% we ask you to refrain from putting more than six authors
% (two rows with three columns) beneath the article title.
% More than six makes the first-page appear very cluttered indeed.
%
% Use the \alignauthor commands to handle the names
% and affiliations for an 'aesthetic maximum' of six authors.
% Add names, affiliations, addresses for
% the seventh etc. author(s) as the argument for the
% \additionalauthors command.
% These 'additional authors' will be output/set for you
% without further effort on your part as the last section in
% the body of your article BEFORE References or any Appendices.

\numberofauthors{2} %  in this sample file, there are a *total*
% of EIGHT authors. SIX appear on the 'first-page' (for formatting
% reasons) and the remaining two appear in the \additionalauthors section.
%
\author{
% You can go ahead and credit any number of authors here,
% e.g. one 'row of three' or two rows (consisting of one row of three
% and a second row of one, two or three).
%
% The command \alignauthor (no curly braces needed) should
% precede each author name, affiliation/snail-mail address and
% e-mail address. Additionally, tag each line of
% affiliation/address with \affaddr, and tag the
% e-mail address with \email.
%
% 1st. author
\alignauthor
Timothy Goodrich\\
\affaddr{Department of Computer Science}\\
\affaddr{North Carolina State University}\\
\affaddr{890 Oval Drive}\\
\affaddr{Raleigh, North Carolina 27606}\\
\email{tdgoodri@ncsu.edu}
% 2nd. author
\alignauthor
Tim Menzies\titlenote{Course Advisor}\\
\affaddr{Department of Computer Science}\\
\affaddr{North Carolina State University}\\
\affaddr{890 Oval Drive}\\
\affaddr{Raleigh, North Carolina 27606}\\
\email{tjmenzie@ncsu.edu}
}

\date{7 December 2016}
% Just remember to make sure that the TOTAL number of authors
% is the number that will appear on the first page PLUS the
% number that will appear in the \additionalauthors section.

\maketitle
\begin{abstract}
This is my abstract.
\end{abstract}

%
% %
% % The code below should be generated by the tool at
% % http://dl.acm.org/ccs.cfm
% % Please copy and paste the code instead of the example below.
% %
% \begin{CCSXML}
% <ccs2012>
%  <concept>
%   <concept_id>10010520.10010553.10010562</concept_id>
%   <concept_desc>Computer systems organization~Embedded systems</concept_desc>
%   <concept_significance>500</concept_significance>
%  </concept>
%  <concept>
%   <concept_id>10010520.10010575.10010755</concept_id>
%   <concept_desc>Computer systems organization~Redundancy</concept_desc>
%   <concept_significance>300</concept_significance>
%  </concept>
%  <concept>
%   <concept_id>10010520.10010553.10010554</concept_id>
%   <concept_desc>Computer systems organization~Robotics</concept_desc>
%   <concept_significance>100</concept_significance>
%  </concept>
%  <concept>
%   <concept_id>10003033.10003083.10003095</concept_id>
%   <concept_desc>Networks~Network reliability</concept_desc>
%   <concept_significance>100</concept_significance>
%  </concept>
% </ccs2012>
% \end{CCSXML}
%
% \ccsdesc[500]{Computer systems organization~Embedded systems}
% \ccsdesc[300]{Computer systems organization~Redundancy}
% \ccsdesc{Computer systems organization~Robotics}
% \ccsdesc[100]{Networks~Network reliability}


%
% End generated code
%

%
%  Use this command to print the description
% %
% \printccsdesc

% We no longer use \terms command
%\terms{Theory}

\keywords{Software Product Lines; Adiabatic Quantum Computing; Multi-Objective Optimization}

\section{Introduction}
\subsection{Motivation}
What is the business-perspective motivation? Why do we care about optimizing Software Product Lines? Why do we care about using the D-Wave quantum annealer?

\subsection{Contributions}
\begin{itemize}
\item \textbf{RQ1 (Conversion Best Practice):} \emph{Which Ising Model formulation algorithm should you use for Softare Product Lines? Compare run time, qubits used, and bipartite proximity.}
\item \textbf{RQ2 (Performance):} \emph{How does D-Wave annealer performance compare to SAT4j? Compare run time and number of optimal solutions found.}
\item \textbf{RQ3 (Scalability):} \emph{How does performance of large optimization problems compare to those small enough to fit on the D-Wave annealer? Compare run time and number of optimal solutions found.}
\end{itemize}

\section{Background}
Introduce the following: Software Product Lines, the specific input corpus we use in this paper (SPLOT for RQ1 and RQ2, LVAT for RQ3), Constrained Satisfiability Problem, Conjunctive Normal Form, SAT4j, the D-Wave annealer, the Ising Model problem.

\section{RQ1: Conversion Best Practice}
In this section we convert our SPLOT problems into Ising Models using the following algorithms:
\begin{enumerate}
\item $\textsc{CSP} \to \textsc{CNF} \to \textsc{IsingModel}$ using boolean logic gadgets.
\item $\textsc{CSP} \to \textsc{MaxIndSet} \to \textsc{IsingModel}$ using a conflict graph.
\item $\textsc{CSP} \to \textsc{IsingModel}$ using a penalty function computed with an LP-solver.
\end{enumerate}

\subsection{Ising Model Formulation Algorithms}
Define each algorithm in detail, provide hooks to the source code.

\subsection{Experimental Results}
Compare the three algorithms on the SPLOT corpus, using the following three objectives: total conversion run time, size of the constructed Ising Model, and how easily we expect this model is to embed into the D-Wave Chimera hardware (e.g. use Odd Cycle Transversal to measure its ``bipartiteness''). Provide plots and a ranking by Scott-Knot.

\subsection{Discussion}
Answer RQ1 by providing the best practice. Provide the following open problems: alternative formulation algorithms we could try (if any exist), different preprocessing techniques we could try (clustering, etc.), and directions for future algorithm design.


\section{RQ2: Empirical Performance}
In this section we embed our SPLOT Ising Models into the physical hardware, run, and compare the run times and solutions found to SAT4j.

\subsection{D-Wave Hardware Details}
Detail how we're embedding the Ising Models (with the API algorithm), what parameters are available and our choice of them, and whether we're using the virtual full-yield Chimera graph or not. Provide a hook to my code's config file, including seeds used, etc.

\subsection{Experimental Results}
Compare the three algorithms and SAT4j on run time, run time with ``compilation time'' (going from the raw CSP to the solution), and the number of valid distinct solutions found. If the data lends itself, maybe provide a solutions-found-per-time-spent curve for each algorithm. Provide plots and a Scott-Knot ranking.

\subsection{Discussion}
Provide conclusions for RQ2: Is D-Wave current competitive as a drop-in replacement for a SAT-solver? Detail the following open problems: parameter optimization with meta-heuristics, and the use of different embedding algorithms.


\section{RQ3: Scalability}
In this section we use an input corpus of much larger SPL datasets, and use 1-2 algorithms to break the problem into pieces. We want to see how the results from the previous sections scale to problems too big for the hardware.

\subsection{Decomposition Algorithms}
Detail one or both of D-Wave's recent divide-and-conquer algorithm and belief propagation algorithm.

\subsection{Experimental Results}
Compare the [three conversion algorithms] $\times$ [the 1-2 decomposition algorithms] with SAT4j. See how the previous run time and solutions-found relations change. Provide plots and a Scott-Knot ranking.

\subsection{Discussion}
Provide a conclusion for RQ3: Does it make sense to use the D-Wave hardware for ``sufficiently large'' SPL problems? Discuss the open problem of ``approximating'' the original CSP by removing choice cross-tree constraints.


\section{Conclusion}

\subsection{Summary of Results}
Summarize the answers to the research questions detailed in the ``Discussion'' subsections at the end of the middle three sections.

\subsection{Validity of Software Science}
Why is this software science and not merely data dabbling?
\begin{itemize}
\item We asked and addressed concrete questions: We identify a best practice for converting SPL datasets into a D-Wave compatible format (RQ1), we see if D-Wave is currently a viable alternative to a SAT solver (RQ2), and see if D-Wave is worth using for large datasets (RQ3).
\item We used publicly-available and previously cited datasets.
\item We provide the output data from the D-Wave annealer (if we can, check with D-Wave).
\item We provide the scripts, configuration files, algorithm implementations for all experiments, along with a tutorial in the github repo on how to use these tools.
\item We provide conclusions that are based on tried-and-true statistical tests (Scott-Knot).
\end{itemize}



\section{Acknowledgments}
To come.

%\bibliographystyle{abbrv}
%\bibliography{bib}  % sigproc.bib is the name of the Bibliography in this case

\end{document}
